\documentclass[a4paper,10pt]{article}
\usepackage{ucs}
\usepackage[utf8]{inputenc}
\usepackage{babel}
\usepackage{fontenc}
\usepackage{graphicx}
\usepackage{cuisine}

\newcommand{\ingredientAmount}[3]{\ingredient[#1]{#2}{#3}}
\newcommand{\recipeStep}[1]{\newstep\noindent\textbf{#1:}}

\begin{document}
    \begin{recipe}{Pizza Margherita}{10 Stück}

    \recipeStep{Vorteig}

    \ingredient[320]{g}{Mehl Tipo 00}
    \ingredient[320]{ml}{Wasser}
    \ingredient[0,1]{g}{Frischefe}

    Lauwarmens Wasser und Hefe in ein verschliessbares Glas geben, vermischen und 5m stehen lassen, um die Hefe zu aktivieren.
    \\

    Mehl dazugeben und vermischen, dann verschliessen und fuer max. 72h bei 5-10\0C ruhen lassen.

    \recipeStep{Hauptteig}

    \ingredient[1280]{g}{Mehl Tipo 00}
    \ingredient[680]{ml}{Wasser}
    \ingredient[50]{g}{Salz}
    \ingredient[20]{ml}{Olivenöl}

    Wasser Mehl und Vorteig aus Schritt 1 vermischen und 5m stehen lassen \\

    Anschliessend das Salz dazugeben und den Teig mit einem Mixer kurz vorkneten, und dann haendisch weiterkneten, indem der Teig kontinuirlich gezogen und gefaltet wird,
    bis dieser fettig aussieht, sich glatt anfühlt, nichtmehr gut dehnbar aber dafür elastisch ist.
    \\

    Den Teig mit einem feuchten Tuch bedeckt ruhen lassen, bis sich eine leichte Kruste bildet(etwa 1h).
    \\

    Mit einer Küchenspachtel den Teig in Portionen von 200-280g (22 bzw.35cm Pizza) teilen.
    \\


    Etwas Olivenöl auf die Teige geben und zu Kugeln formen, in dem gestopft und abgezwickt wird (Mozzatura).
    \\

    Die Teigkugeln in einer geschlossenen Plastikbox bis zu 24h bei 20,5-25.3°C ruhen lassen.\\


    \recipeStep{Belag vorbereiten}


    \ingredient[600]{g}{Toamten(San Marzano)}
    \ingredient[60]{ml}{Olivenöl}
    \ingredient[ein paar]{Blätter}{frisches Basilikum}
    \ingredient[700]{g}{Büffelmozzarella oder Fior die Latte}


    Für die Souce Tomaten, Olivenöl, Salz, sowie einen Teil des Basilikums in eine Schüssel geben, mit der Hand zerdrücken und gut vermischen.
    \\

    Mozzarella in 1x1cm Streifen schneiden und im Kühlschrank trocknen lassen.

    \recipeStep{Pizza formen}
    \ingredient[100]{g}{Weichweizengrieß}

    Die fertig gereiften Teiglinge aus der Box nehmen und auf einer mit Weichweizengrieß leicht bemehlten Arbeitsfläche zu einer Scheibe formen,
    dazu wird der Teig mit den Fingern von innen nach außen gearbeitet, und mehrfach gewendet, bis in der Mitte ca 1mm und am Rand ca. 1cm erreicht werden


    \newstep
    \ingredient[50]{g}{Parmesan}
    \ingredient[ein paar]{Blätter}{frisches Basilikum}

    Die Tomatensouce mit einem Löffel auf dem Teig verteilen, Basilikumblätter, Mozzarella Streifen darauf verteilen, und Parmesan grob darüber reiben.
    \\
    \recipeStep{Pizza backen}

    Die Pizza mit einer Schaufel in einen Ofen mit 485\0C bei 430\0C Oberflächentemperatur fuer 60-90s backen.



    \end{recipe}
\end{document}
